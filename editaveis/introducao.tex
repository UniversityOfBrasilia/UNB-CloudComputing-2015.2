\chapter[Introdução]{Introdução}\label{cap1}

Este artigo é uma proposta de trabalho prático da disciplina de Computação em Nuvem, ministrada na Universidade de Brasília - Faculdade do Gama. O trabalho propõe implementar um artigo, utilizando uma ferramenta de computação em nuvem. Aqui foi implementado o artigo Estratégia para Alocação Dinâmica de Recursos em um Ambiente Híbrido de Computação em Nuvem, utilizando o software Openstack.

A computação em nuvem busca deslocar toda infraestrutura computacional para a rede, assim o custo de software e hardware podem ser reduzidos. Existem quatro modelos de implementação na computação em nuvem: a nuvem privada, pública, comunitária e híbrida.

Sua arquitetura possui   três camadas a de infraestrutura, plataforma e aplicação. A Figura 1, mostra a arquitetura da computação em nuvem (Zhang et al, 2010). Openstack é um exemplo de serviço da camada IaaS.

imagem Figura 1 - Arquitetura da Computação em Nuvem.
