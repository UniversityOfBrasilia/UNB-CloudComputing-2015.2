\chapter[Artigo Implementado]{rtigo Implementado}\label{cap1}

O artigo Estratégia para Alocação Dinâmica de Recursos em um Ambiente Híbrido de Computação em Nuvem, procurou trabalhar na parte de prover recursos sob demanda para garantir a qualidade como serviço. Para isso foi proposta uma solução autonômica, que modifica dinamicamente a quantidade de recursos de CPU  e instancia/remove Máquinas Virtuais.  Para obtenção dos resultados, foram realizados experimentos em uma nuvem híbrida, com um balanceador de carga que distribui as requisições de uma aplicação web, a qual foi alinhada a diferentes estratégias para alocação dinâmica de recursos.

A solução proposta, adiciona a arquitetura FairCPU, que utiliza o conceito de Unidade de Processamento para alocar recursos de CPU a máquinas virtuais, o conceito de sensor e loop de controle, que fornece a capacidade de computação autonômica.

A Figura 2 mostra o elemento básico do loop de controle.


imagem Figura 2 - Loop de Controle.


Na Figura 3, é possível visualizar a solução proposta, e sua relação com a arquitetura FairCPU. A abordagem foi dividida em duas partes: Agente e Controlador Autonômico.


imagem vFigura 3 - Visão geral da solução proposta.

Para testar a solução foram realizados três experimentos, executados em dois ambientes, uma nuvem privada construída com o OpenNebula e uma nuvem pública da Amazon.
